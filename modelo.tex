{	%	---------------------------- Apresentacao -----------------------------------	%
	%																					%
	%	Desenvolvendo um modelo padrao para trabalhos academicos utilizando o ABNTeX2	%
	%	Autor: Salomão Luiz de Araújo Neto												%
	%	GitHub:	https://github.com/salomaoluiz											%
	%	LinkedIn: https://www.linkedin.com/salomao-luiz									%
	%																					%
	%	-----------------------------------------------------------------------------	%	
}

{	%	------------------------ Informacoes de Uso ---------------------------------	%
	%																					%
	%	1. Nao e utilizado acentuacao nas areas de comentario para evitar crash de	 	%
	%	texto ao ser utilizado em outro sistema, tornando os comentarios ilegiveis.		%
	%																					%
	%	2. Ao utilizar este documento, por gentileza de creditos ao autor, com isto		%
	%	o seu e o meu trabalho sao valorizados.											%
	%																					%
	%	3. Realize as modificacoes que achar necessario, seja para adequar aos pedidos	%
	%	de sua universidade, ou por gosto proprio, mas caso queira realizar uma 		%
	%	contribuicao com o codigo, o mantenha sempre comentado, para um melhor			%			
	%	entendimento.																	%
	%																					%	
	%	4. Softwares Utilizados: TeXstudio e MikTeX, sempre que for compilar seu		%
	%	projeto, tenha certeza de possuir conexao com a internet, para que seja 		%
	%	possivel o download dos pacotes necessarios.									%
	%																					%
	%	5. A tela utilzada para a elaboracao deste projeto foi uma 1366x768, recomendo	%
	%	a utilizacao de uma de igual ou maior resolucao para nao haver bagunca no		%
	%	desing do projeto.																%
	%	-----------------------------------------------------------------------------	%
}

	%	-----------------------------------------------------------------------------	%
	%	---------------------------	Inicio do Preambulo	-----------------------------	%
	%	-----------------------------------------------------------------------------	%

\documentclass[	12pt,								%Tamanho da Fonte
				openright,							%Abre a Direita
				twoside,							%Imprime Frente/Verso
				a4paper,							%Folha A4
				chapter = TITLE,					%Capitulos em Maiusculo
				section = TITLE,					%Secoes em Maiusculo
				subsection = TITLE,					%Subsecoes em Maiusculo
				subsubsection = TITLE,				%Subsubsecoes em Maiusculo
				english, french, spanish, brazil]	%Idiomas Usados
				{abntex2}

	%	-----------------------------------------------------------------------------	%
	%	---------------------------	Pacotes Utilizados	-----------------------------	%
	%	-----------------------------------------------------------------------------	%

	\usepackage{lmodern}					% Define a fonte Latin Modern para o documento
	\usepackage[T1]{fontenc}				% Para a utilizacao de acentuacao
	\usepackage[utf8]{inputenc}				% Utilizacao do codificador UTF8
	\usepackage{listings, verbatim}			% Insercao de codigos de programacao
	\usepackage{indentfirst}				% Identa o primeiro paragrafo de cada secao
	\usepackage{babel}						% Utilizado para Renomear as Secoes
	\usepackage{lastpage}					% Utilizado para Pegar a Ultima Folha
	\usepackage{graphicx}					% Insercao de Figuras
	\usepackage{amsmath}					% Permite o uso de algumas funcoes matematicas
	\usepackage[alf]{abntex2cite}			% Padrao de Citacao ABNT
	\usepackage{xcolor, lipsum}				% Adiciona novas Cores


	
	%	-----------------------------------------------------------------------------	%
	%	-------------	Configurações de Linguagens de Programacao ------------------	%
	%	-----------------------------------------------------------------------------	%
	

	%	-----------------------------------------------------------------------------	%
	%	-----------------------	Configurações das Paginas	-------------------------	%
	%	-----------------------------------------------------------------------------	%
	
	%	----------------- Define Fonte e tamanho dos Capitulos e Secoes -------------	%
	\renewcommand{\ABNTEXchapterfont}{\fontfamily{ptm}\fontseries{b}\selectfont}
	\renewcommand{\ABNTEXchapterfontsize}{\Large}
	\renewcommand{\ABNTEXsectionfontsize}{\normalsize}
	\renewcommand{\footnotesize}{\small}
	\renewcommand{\ABNTEXsubsectionfontsize}{\normalsize}
	%-----------------------------------------------------------------------------------%
	
	%	----------------- Define Tamanho e Altura entre Paragrafos ------------------	%	
	\setlength{\parindent}{1.3cm}		% Define o tamanho do paragrafo
	\setlength{\parskip}{0.2cm}			% Define a altura entre os paragrafos
		
	%	-----------------------------------------------------------------------------	%
	%	-----------------------	Definindo Novos Comandos ----------------------------	%
	%	-----------------------------------------------------------------------------	%
	
	%	------------------- Comandos para a Utilização no Texto ---------------------	%		
	%\renewcommand{Novo Comando}{Comando Antigo}
	\newcommand{\rarrow}{\rightarrow}
	\newcommand{\larrow}{\leftarrow}
	\newcommand{\ba}{$\backslash$}

	
	%	--------------------- Alterações em Nomes Padronizados ----------------------	%		
	\addto\captionsbrazil
	{
		\renewcommand{\bibname}{Referências}
		\renewcommand{\indexname}{Índice}
		\renewcommand{\listfigurename}{Lista de Ilustrações}
		\renewcommand{\listtablename}{Lista de Tabelas}
		\renewcommand{\pageautorefname}{Página}
		\renewcommand{\sectionautorefname}{Seção}
		\renewcommand{\subsectionautorefname}{Subseção}
		\renewcommand{\paragraphautorefname}{Parágrafo}
		\renewcommand{\subsubsectionautorefname}{Subsubseção}
	}
	%	-----------------------------------------------------------------------------	%
	%	-----------------------	Configurar Desing da Capa ---------------------------	%
	%	-----------------------------------------------------------------------------	%
%	\renewcommand{\imprimircapa} 
%	{ 
%		\begin{capa}
%			\center		
%			{\ABNTEXchapterfont\normalsize	\imprimirinstituicao}
%			
%			\vspace*{1cm}
%			
%			{\ABNTEXchapterfont\mdseries\large	\imprimirautor}
%			
%			\vspace*{4cm}
%			{\ABNTEXchapterfont\Large	\imprimirtitulo}
%			
%		\end{capa}
%	}
	
	
	
	%	-----------------------------------------------------------------------------	%
	%	-----------------------	Informacoes do Trabalho -----------------------------	%
	%	-----------------------------------------------------------------------------	%
	
	\titulo{Desenvolvendo Trabalhos Acadêmicos com a Utilização da Plataforma LaTeX, Utilizando a Classe AbnTeX2}
	\autor{Salomão Luiz de Araújo Neto}
	\instituicao{UNIVERSIDADE DO ESTADO DO MATO GROSSO - UNEMAT \par
				 FACULDADE DE CIÊNCIAS EXATAS E TECNOLÓGICAS	\par
			 	 DEPARTAMENTO DE MATEMÁTICA						\par
		 	 	 Campus de Sinop - MT}
	\local{Sinop, MT}
	\data{2018}
	\tipotrabalho{Monografia}
	\orientador[Profº Dr. ]{Nome Orientador}
	
	\preambulo{Projeto de Pesquisa apresentado à Banca Examinadora do Curso de ** da Universidade **, Campus de **, como parte dos requisitos para obtenção do título de **.}
	
\begin{document}

	%	--------------------- Elementos Pre-Textuais ----------------------	%
\imprimircapa
\imprimirfolhaderosto*{}
\begin{fichacatalografica}
	\sffamily
	\vspace*{15cm}		%Posicao Vertical

	\begin{center}
		INSERIR FIGURA DA FICHA CATALOGRÁFICA AQUI
	\end{center}
\end{fichacatalografica}
%\input{./elementosPreTextuais/errata/errata.tex}
%	-------------- Utilizar Isto Quando conseguir a folha de Aprovacao --------------	%
%\includepdf{./arquivos/folhaAprovacaoFinal.pdf}

%	---------------------- Utilizar Isto de Forma Temporaria -----------------------	%
\begin{folhadeaprovacao}
	\begin{center}
		{\ABNTEXchapterfont\large\imprimirautor}
		\vspace*{\fill}\vspace*{\fill}
		\begin{center}
			\ABNTEXchapterfont\bfseries\Large\imprimirtitulo
		\end{center}
		\vspace*{\fill}
		\hspace{.45\textwidth}
		\begin{minipage}{.5\textwidth}
			\imprimirpreambulo
		\end{minipage}%
		\vspace*{\fill}
	\end{center}
	Trabalho aprovado. \imprimirlocal, 24 de novembro de 2012:
	\assinatura{\textbf{\imprimirorientador} \\ Orientador}
	\assinatura{\textbf{Professor} \\ Convidado 1}
	\assinatura{\textbf{Professor} \\ Convidado 2}
	\begin{center}
		\vspace*{0.5cm}
		{\large\imprimirlocal}
		\par
		{\large\imprimirdata}
		\vspace*{1cm}
	\end{center}
\end{folhadeaprovacao}
\begin{dedicatoria}

\textodedicatoria{Agradeço ao mundo por sempre evoluir ao seu tempo e sua maneira, pois assim não teríamos o que pesquisar, o que descobrir e o que nos motivar a viver. Por seus mistérios, ainda não desvendados e pelas pessoas que habitam nele, pois através disto consegui ter considerações finais sobre muitos conceitos, não somente deste trabalho, mas para futuras "diversões"}

\imprimirdedicatoria
\end{dedicatoria}
\begin{agradecimentos}

Agradeço à...

\end{agradecimentos}
\begin{epigrafe}
	
\textoepigrafe{Não há culpa maior / do que entregar-se às vontades / não há mal maior / do que aquele de não saber contentar-se / não há dano maior / do que nutrir o desejo de conquista.}

\autorepigrafe{Lao-Tsé}

\imprimirepigrafe
\end{epigrafe}
\begin{resumo}

Resumo em Português

\vspace{\onelineskip}
\noindent
\textbf{Palavras-chave}: latex. abntex. editoração de texto.

\end{resumo}
\begin{resumo}[Abstract]
	\begin{otherlanguage*}{English}
		Abstract in English
		
		\vspace{\onelineskip}
		\noindent
		\textbf{Keywords}: latex. abntex. text publisher.
		
	\end{otherlanguage*}
\end{resumo}
\input{./elementosPreTextuais/resumen/resumen.tex}

\listoffigures*
\cleardoublepage
\listoftables*
\cleardoublepage

\begin{siglas}
	\item[ABS] Acrylonitrile Butadiene Styrene
	\item[PLA] Ácido Polilático
\end{siglas}
\begin{simbolos}
	\item[ºC] Graus Célcios
	\item[ºF] Graus Fahrenheit
	\item[$\gamma$] Letra Grega Gamma
	\item[$\Lambda$] Letra Grega Lambda
\end{simbolos}
\tableofcontents*

	%	--------------------- Elementos Textuais -----------------------	%

\mainmatter
\chapter*{Introdução}
\addcontentsline{toc}{chapter}{INTRODUÇÃO}

\chapter{Softwares Utilizados}

\section{MikTeX}
\section{TeXstudio}
\subsection{Principais Características do TeXstudio}
\chapter{Configurações do Preâmbulo}
\section{Configurações do \textbackslash documentclass}
O \lstinline|\documentclass{}| é uma das configurações básicas de um documento \LaTeX, nele é onde é definido as principais características do documento, como se sera um trabalho acadêmico, um artigo, banner ou slide. Por ele é possível configurar os padrões do documento, como tamanho de fonte ou da folha, a linguagem, configurações de titulo e subtítulos.

As configurações utilizadas no \lstinline|\documentclass{}| foram:
\begin{itemize}
	\item 12pt $\rarrow$ Tamanho de Fonte 12												
	\item openright $\rarrow$ Inicia a pagina pela direita										
	\item twoside $\rarrow$ A pagina sera impressa frente e verso
	\item a4paper $\rarrow$ O papel padrão com tamanho A4									
	\item brazil $\rarrow$ Define a linguagem com Português-Brasil							
	\item abntex2 $\rarrow$ Para utilizar a classe do documento com normas ABNT				
\end{itemize}


\chapter{Etapa Textual}
\section{Capítulos, Seções e Subseções}

Os capítulos podem ser criados usando o comando \lstinline|\chapter{}| isto irá criar um capítulo como os destes trabalho, conforme as normas, para seções e subseções tem os seguintes códigos:
\begin{itemize}
	\item \lstinline|\section{}|
	\item \lstinline|\subsection{}|
	\item \lstinline|\subsubsection{}|
\end{itemize}

para que o capítulo ou seção não seja numerada, como por exemplo a Introdução, basta adicionar um * logo após o comando, mas antes do texto, por exemplo: \lstinline|\chapter*{}|.

\section{Equações e Simbologia Matemática}
Apesar da plataforma de desenvolvimento LaTeX poder ser utilizado por qualquer ramo da ciência para o desenvolvimento de trabalhos com uma excelente tipografia, ela é principalmente utilizada por pessoas das áreas exatas, por conta da enorme facilidade em desenvolver trabalhos com enormes quantidades de equações e fórmulas, com estas se mantendo sempre organizadas. Para fazer a inserção de equações, funções ou simbologia matemática é preciso estar dentro do ambiente matemático, este é uma área apenas para a inserção de fórmulas matemáticas. 

Este espaço pode ser feito de duas formas, dentro do texto, ou em um ambiente separado. Para a utilização dentro do texto, as equações matemáticas precisam ser inseridas dentro de um par de \$, por exemplo \$ 3x\textasciicircum 2 = 2 \$, ao usar essa forma, o código fica da seguinte maneira: $3x^2=2$.

Outra forma de se adicionar fórmulas, funções ou símbolos matemáticos é pelo ambiente \lstinline[language=TeX]|\begin{equation} \end{equation}| tudo que se colocar dentro destas funções será centralizado, enumerado, e ficará em formato matemático. Por exemplo, ao usar:

\begin{figure}[htb]
	\begin{center}
		\includegraphics[scale=1]{./Imagens/capitulo_2/code_1.png}
	\end{center}
\end{figure}

tem-se como resultado:

\begin{equation}\label{equacao1}
	3x^2 = 2
\end{equation}

Está forma é muito útil quando você quer referenciar alguma equação ao longo do texto, pois com o comando \lstinline[language=TeX]|\label{}| você pode em qualquer local utilizar o \lstinline[language=TeX]|\ref{}| para referenciar aquele ``label'', por exemplo: ``Seja a equação \ref{equacao1} tem-se que...'' é possível fazer isso para todas as equações dentro do ambiente ``equation'', figuras, tabelas, basta alterar o que está escrito dentro do ``label'', e usar o ``ref'' para referenciar ela.

Outra forma de inserir equações matemáticas é utilizando \lstinline[language=TeX]|\[ \]| o que colocar dentro destes colchetes será centralizado, mas não sera enumerado, por exemplo:

\begin{figure}[htb]
	\begin{center}
		\includegraphics[scale=1.5]{./Imagens/capitulo_2/code_2.png}
	\end{center}
\end{figure}

terá como resultado:
\[
3x^2=2
\]
\subsection{Simbologia Matemática}
O LaTeX possui suporte a diversos símbolos matemáticos, desde simbologia como $\pm \div \bullet \bigtriangleup \neq \gg$ como também letras gregas como $\beta \gamma \delta \epsilon \varepsilon$ entre muitos outros. Para ver todos os símbolos matemáticos vá em ``View'' em ``Show'' e selecione ``Side Panel'', com isso irá abrir uma tela no canto lateral, navegue por ela e veja todos os símbolos que se pode adicionar, sempre fique atento em colocar os símbolos dentro de um ambiente matemático, se não será impossível compilar o projeto.
\subsection{Trabalhando com Equações}
O LaTeX tem suporte a diversas funções matemáticas, e alguns comandos que possibilitam o melhoramento dessas equações, a tabela

\begin{table}[htb]
	\IBGEtab{%
		\caption{Algumas Funções Matemáticas}%
		\label{tab_Cap2_equacoes}
	}{%
	\begin{tabular}{ccc|ccc}
		\toprule
		     Equação      &                   Código                    &      Resultado       &       Equação       &                Código                &     Resultado     \\ \midrule\midrule
		  Raiz Quadrada   &                \ba sqrt\{x\}                &      $\sqrt{x}$      & Integral Indefinida &             \ba int\{x\}             &     $\int{x}$     \\
		Raiz a Potencia N &              \ba sqrt[3]\{x\}               &    $\sqrt[3]{x}$     &  Integral Definida  & \ba int\_{2}\textasciicircum{3}\{x\} & $\int_{2}^{3}{x}$ \\
		    Somatoria     & \ba sum\_\{i=1\}\textasciicircum\{10\}\{x\} & $\sum_{i=1}^{10}{x}$ &       Fração        &          \ba frac\{x\}\{y\}          &   $\frac{x}{y}$   \\ \bottomrule
	\end{tabular}%
}{%
\fonte{Autoria Própria}%
}
\end{table}

Muitas outras funções podem ser obtidas indo em ``Math'' e em ``Math Function''. Observe que quando se utiliza uma função complexa dentro de uma tabela, como por exemplo a somatória, ela fica com a aparência um pouco ruim, com pouca organização, para resolver este problema, basta colocar antes da equação o comando \lstinline[language=TeX]|\displaystyle| assim ela ficará da seguinte forma:
\[
\sum_{i=1}^{10}{x}
\]

O mesmo vale para integrais, frações, raízes, entre outras.
\subsection{Tabulações}
\subsubsection{Matrizes}
Para a criação de matrizes existem os ambientes ``pmatrix'', ``bmatrix'', ``vmatrix'', ``Vmatrix'', ``matrix'' e ``array''. O ``pmatrix'' serve para a criação de matrizes com parenteses nas bordas, como por exemplo:
\[
\begin{pmatrix}
	2x & 3x \\ 
	x & 4x
\end{pmatrix} 
\]

o ``bmatrix'' serve para criar matrizes na forma de caixa, por exemplo:
\[
\begin{bmatrix}
2x & 3x \\ 
x & 4x
\end{bmatrix} 
\]

o ``vmatrix'' e o ``Vmatrix'' servem para criar matrizes com barras nas bordas, o primeiro com 1 barra, e o segundo com 2 barras, por exemplo: \\
\begin{center}
$
\begin{vmatrix}
2x & 3x \\ 
x & 4x
\end{vmatrix} 
$
\hspace*{2cm}
$
\begin{Vmatrix}
2x & 3x \\ 
x & 4x
\end{Vmatrix} 
$
\end{center}

o ``matrix'' cria uma matriz sem nenhuma borda
\[
\begin{matrix}
2x & 3x \\ 
x & 4x
\end{matrix}
\]

já o ``array'' é possível editar para colocar barras entre cada coluna ou linha, por exemplo:
\[
\begin{array}{c|ccc}
2x & 3y &2z & 4w\\ 
x & 4y & 3z & 8w \\ \hline
3x & 7y & 5z & 12w
\end{array}
\]
é possível também colocar o array dentro de um ``bmatrix'', ou ``pmatrix'' ou qualquer outra matriz, mas caso você deseje utilizar por exemplo, um lado de parentese e outro de colchete, é preciso usar as funções \lstinline|\left| e \lstinline|\right| acompanhado do simbolo que deseje, por exemplo, o código a seguir, apresenta o seguinte resultado:

\begin{figure}[htb]
	\begin{center}
		\includegraphics[scale=1]{./Imagens/capitulo_2/code_3.png}
	\end{center}
\end{figure}

\[
\left[
\begin{array}{c|ccc}
2x & 3y & 2z & 4w  \\
x  & 4y & 3z & 8w  \\ \hline
3x & 7y & 5z & 12w
\end{array}
\right)
\]

\subsubsection{Tabelas}
Para a utilização de Tabelas nas normas da ABNT é preciso usar os comandos:

\begin{figure}[htb]
	\begin{center}
		\includegraphics[scale=1]{./Imagens/capitulo_2/code_4.png}
	\end{center}
\end{figure}

Isto fornece a seguinte tabela:

\begin{table}[htb]
\IBGEtab{%
\caption{Aqui vai o titulo da tabela}%
\label{tab_Cap2_exemplo}
}{%
\begin{tabular}{ccc}
	\toprule
	     Nome       &     Cidade     &    Estado    \\ \midrule\midrule
	Pedro da Silva  &     Cuiaba     & Mato Grosso  \\
	  João Neves    &   São Paulo    &  São Paulo   \\
	Maria Antonieta & Belo Horizonte & Minas Gerais \\ \bottomrule
\end{tabular}%
}{%
\fonte{Autoria Própria}%
\nota[Nota1]{Aqui pode ser inserido uma nota sobre a tabela}
}
\end{table}


\subsection{Apresentação de Códigos de Programação}
Para a apresentação de códigos de programação, utiliza-se do pacote listings, ele é responsável por apresentar código em diferentes linguagens, com as fontes e detalhes específicos de cada linguagem, com destaques em suas palavras-chave ou em seus comentários. Para mudar as cores e outros detalhes da linguagem, olhe a parte ``Configurações de Linguagens de Programacao'' do preâmbulo.

Para utilizar este pacote, adicione-o inicialmente no preâmbulo e já será possível usa-lo através do ambiente lstlisting. Por exemplo:
\begin{lstlisting}[language = Java]
public main Codigo(){
 public void main[String args]{
  int calcular = 0;
  for(int i = 0; i <= 10; i++){
   calcular += i;
   System.out.printf("Ola Mundo! O valor: ", calcular);
  }
 }
}
\end{lstlisting}

Está normatização pode ser feita para quase todas as linguagens, algumas eu não tive sucesso em conseguir configurar, um exemplo é a própria linguagem TeX, que não tive sucesso na configuração. 
\section{Inserção de Imagens}
Para adicionar imagens, conforme as normas da ABNT, é necessário utilizar o código:

\begin{figure}[htb]
	\begin{center}
		\includegraphics[scale=1]{./Imagens/capitulo_2/code_5.png}
	\end{center}
\end{figure}

Com este código, a imagem fica na seguinte forma:

\begin{figure}[htb]			
	\caption{Brasão Unemat \label{fig_Cap2_brasaoUnemat}}
	\begin{center}
		\includegraphics[scale=0.1]{./Imagens/Brasao_Unemat.png}
	\end{center}
	\legend{Fonte: http://sinop.unemat.br/site/}
\end{figure}
%\chapter{Elementos Pós Textuais}
\section{Bibliografia}

Para criar um bibliografia, vá na pasta ``bibliografia'' dentro da pasta ``elementosPosTextuais'' e edite o arquivo. Existem vários tipos de bibliografias que se podem utilizar, como por exemplo:

\begin{table}[htb]
	\IBGEtab{%
		\caption{Tipos de Bibliografias}%
		\label{tab_Cap3_bibliografia}
	}{%
		\begin{tabular}{ccc}
			\toprule
			   Tipo    &     Tipo      &     Tipo      \\ \midrule\midrule
			 article   &     book      &    manual     \\
			   www     &    booklet    &   commented   \\
			  inbook   & incollection  & inproceedings \\
			jurthesis  & mastersthesis &     misc      \\
			periodical &   phdthesis   &  proceedings  \\
			techreport &  unpublished  &               \\ \bottomrule
		\end{tabular}%
	}{%
		\fonte{\cite{ferreira}}%
	}
\end{table}

\subsection{Criando uma Referência}
Para criar uma referência, é preciso ir até o arquivo da bibliografia, e dependendo do tipo de referência, alguns elementos precisam ser colocados, por exemplo, ao usar um artigo de referência é preciso colocar:
\begin{lstlisting}
@article{nomeReferencia,
 title = {Titulo do Artigo},
 author = {Autor do Artigo},
 year = {Ano de Publicacao},
 jounal = {Publicadora},
 address = {Local de Publicacao}
}
\end{lstlisting}

Isso irá criar uma referencia, que poderá ser chamada em uma citação utilizando de 3 formas, citações diretas/indiretas curtas e longas, e  citações no texto.

Uma observação quanto a criação de referência, é que em muitas vezes o software não consegue reconhecer acentuações, então é preciso utilizar de comandos para inserir os acentos, os principais comandos são:
\begin{table}[htb]
	\IBGEtab{%
		\caption{Acentuação}%
		\label{tab_Cap3_acentos}
	}{%
		\begin{tabular}{cc}
			\toprule
			         Comando          & Acento \\ \midrule\midrule
			        \ba'\{a\}         &   á    \\
			       \ba ~\{a\}         &   ã    \\
			       \ba `\{a\}         &   à    \\
			\ba\textasciicircum \{a\} &   â    \\
			       \ba c\{c\}         &   ç    \\ \bottomrule
		\end{tabular}%
	}{%
		\fonte{Autoria Própria}%
	}
\end{table}
\subsection{Formas de Citações}
Uma citação no texto pode ser usada com o comando \lstinline|\citeonline{nomeReferencia}|. Por exemplo: ``Conforme dito por \citeonline{gajanan} tem-se que...''.

Outra forma de se fazer uma citação é de forma direta curta, como foi utilizado na tabela anterior, para fazer uma citação dessa forma, utiliza-se após a citação o comando \lstinline|\cite{nomeReferencia}|. Por exemplo: ``Existem centenas de estilos bibliográficos mundo a fora.''\cite{araujo2016}

Para citações grandes, com mais de 3 linhas, é preciso utilizar um ambiente especifico para citações longas \lstinline|\begin{citacao} Texto \cite{nomeReferencia}\end{citacao}|. Utilizando deste ambiente, é possível fazer citações da forma:
\begin{citacao}
	Três anos depois de ter anunciado uma descoberta há muito esperada pelos físicos, o bóson de Higgs, a Organização Europeia para a Pesquisa Nuclear (CERN, na sigla em francês) divulga a melhor representação da partícula já capturada até hoje. A imagem, apresentada nesta terça-feira (01/09/2015) durante uma conferência anual da instituição, foi o resultado da combinação dos dados coletados no Grande Colisor de Hádrons por dois experimentos diferentes, o ATLAS e o CMS, entre os anos de 2011 e 2012. \cite{oliveira2015}
\end{citacao}

Exite ainda uma 4º forma de realizar uma citação, mas neste caso, não existe citação em sí, apenas a inserção da referência na lista de bibliografia. Não recomendo utilizar está forma, mas caso seja necessário utilize o comando \lstinline|\nocite{nomeReferencia}|. Por exemplo, irei citar o livro Teorias de Aprendizagem de Marco Antônio Moreira sem apresentar nenhuma citação, apenas adicionando o comando \lstinline|\nocite{moreira2011}|. \nocite{moreira2011}



	%	------------------- Elementos Pós Textuais ---------------------	%
%\bibliography{./elementosPosTextuais/bibliografia/bibliografia}

\end{document}