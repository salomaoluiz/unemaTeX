\chapter{Etapa Textual}

\section{Equações e Simbologia Matemática}
Apesar da plataforma de desenvolvimento LaTeX poder ser utilizado por qualquer ramo da ciência para o desenvolvimento de trabalhos com uma excelente tipografia, ela é principalmente utilizada por pessoas das áreas exatas, por conta da enorme facilidade em desenvolver trabalhos com enormes quantidades de equações e fórmulas, com estas se mantendo sempre organizadas. Para fazer a inserção de equações, funções ou simbologia matemática é preciso estar dentro do ambiente matemático, este é uma área apenas para a inserção de fórmulas matemáticas. 

Este espaço pode ser feito de duas formas, dentro do texto, ou em um ambiente separado. Para a utilização dentro do texto, as equações matemáticas precisam ser inseridas dentro de um par de \$, por exemplo \$ 3x\textasciicircum 2 = 2 \$, ao usar essa forma, o código fica da seguinte maneira: $3x^2=2$.

Outra forma de se adicionar fórmulas, funções ou símbolos matemáticos é pelo ambiente \lstinline|\begin{equation} \end{equation}| tudo que se colocar dentro destas funções será centralizado, enumerado, e ficará em formato matemático. Por exemplo, ao usar:
\begin{lstlisting}
\begin{equations}\label{equacao1}
	3x^2 = 2
\end{equations}
\end{lstlisting}
tem-se como resultado:

\begin{equation}\label{equacao1}
	3x^2 = 2
\end{equation}
\
Está forma é muito útil quando você quer referenciar alguma equação ao longo do texto, pois com o comando \lstinline|\label{}| você pode em qualquer local utilizar o \lstinline|\ref{}| para referenciar aquele ``label'', por exemplo: ``Seja a equação \ref{equacao1} tem-se que...'' é possível fazer isso para todas as equações dentro do ambiente ``equation'', figuras, tabelas, basta alterar o que está escrito dentro do ``label'', e usar o ``ref'' para referenciar ela.

Outra forma de inserir equações matemáticas é utilizando \lstinline|\[ \]| o que colocar dentro destes colchetes será centralizado, mas não sera enumerado, por exemplo:
\begin{lstlisting}
\[
3x^2=2
\]
\end{lstlisting}
terá como resultado:
\[
3x^2=2
\]
\subsection{Simbologia Matemática}
O LaTeX possui suporte a diversos símbolos matemáticos, desde simbologia como $\pm \div \bullet \bigtriangleup \neq \gg$ como também letras gregas como $\beta \gamma \delta \epsilon \varepsilon$ entre muitos outros. Para ver todos os símbolos matemáticos vá em ``View'' em ``Show'' e selecione ``Side Panel'', com isso irá abrir uma tela no canto lateral, navegue por ela e veja todos os símbolos que se pode adicionar, sempre fique atento em colocar os símbolos dentro de um ambiente matemático, se não será impossível compilar o projeto.
\subsection{Trabalhando com Equações}
O LaTeX tem suporte a diversas funções matemáticas, e alguns comandos que possibilitam o melhoramento dessas equações, a tabela

\begin{table}[htb]
	\IBGEtab{%
		\caption{Algumas Funções Matemáticas}%
		\label{tab_Cap2_equacoes}
	}{%
	\begin{tabular}{ccc|ccc}
		\toprule
		     Equação      &                   Código                    &      Resultado       &       Equação       &                Código                &     Resultado     \\ \midrule\midrule
		  Raiz Quadrada   &                \ba sqrt\{x\}                &      $\sqrt{x}$      & Integral Indefinida &             \ba int\{x\}             &     $\int{x}$     \\
		Raiz a Potencia N &              \ba sqrt[3]\{x\}               &    $\sqrt[3]{x}$     &  Integral Definida  & \ba int\_{2}\textasciicircum{3}\{x\} & $\int_{2}^{3}{x}$ \\
		    Somatoria     & \ba sum\_\{i=1\}\textasciicircum\{10\}\{x\} & $\sum_{i=1}^{10}{x}$ &       Fração        &          \ba frac\{x\}\{y\}          &   $\frac{x}{y}$   \\ \bottomrule
	\end{tabular}%
}{%
\fonte{Autoria Própria}%
}
\end{table}

Muitas outras funções podem ser obtidas indo em ``Math'' e em ``Math Function''. Observe que quando se utiliza uma função complexa dentro de uma tabela, como por exemplo a somatória, ela fica com a aparência um pouco ruim, com pouca organização, para resolver este problema, basta colocar antes da equação o comando \lstinline|\displaystyle| assim ela ficará da seguinte forma:
\[
\sum_{i=1}^{10}{x}
\]

O mesmo vale para integrais, frações, raízes, entre outras.
\subsection{Tabulações}
\subsubsection{Matrizes}
Para a criação de matrizes existem os ambientes ``pmatrix'', ``bmatrix'', ``vmatrix'', ``Vmatrix'', ``matrix'' e ``array''. O ``pmatrix'' serve para a criação de matrizes com parenteses nas bordas, como por exemplo:
\[
\begin{pmatrix}
	2x & 3x \\ 
	x & 4x
\end{pmatrix} 
\]

o ``bmatrix'' serve para criar matrizes na forma de caixa, por exemplo:
\[
\begin{bmatrix}
2x & 3x \\ 
x & 4x
\end{bmatrix} 
\]

o ``vmatrix'' e o ``Vmatrix'' servem para criar matrizes com barras nas bordas, o primeiro com 1 barra, e o segundo com 2 barras, por exemplo: \\
\begin{center}
$
\begin{vmatrix}
2x & 3x \\ 
x & 4x
\end{vmatrix} 
$
\hspace*{2cm}
$
\begin{Vmatrix}
2x & 3x \\ 
x & 4x
\end{Vmatrix} 
$
\end{center}

o ``matrix'' cria uma matriz sem nenhuma borda
\[
\begin{matrix}
2x & 3x \\ 
x & 4x
\end{matrix}
\]

já o ``array'' é possível editar para colocar barras entre cada coluna ou linha, por exemplo:
\[
\begin{array}{c|ccc}
2x & 3y &2z & 4w\\ 
x & 4y & 3z & 8w \\ \hline
3x & 7y & 5z & 12w
\end{array}
\]
é possível também colocar o array dentro de um ``bmatrix'', ou ``pmatrix'' ou qualquer outra matriz, mas caso você deseje utilizar por exemplo, um lado de parentese e outro de colchete, é preciso usar as funções \lstinline|\left| e \lstinline|\right| acompanhado do simbolo que deseje, por exemplo:
\begin{lstlisting}
\[
 \left[
  \begin{array}{c|ccc}
  2x & 3y &2z & 4w\\ 
  x & 4y & 3z & 8w \\ \hline
  3x & 7y & 5z & 12w
  \end{array}
 \right)
\]
\end{lstlisting}
que fornece como resultado:
\[
\left[
\begin{array}{c|ccc}
2x & 3y &2z & 4w\\ 
x & 4y & 3z & 8w \\ \hline
3x & 7y & 5z & 12w
\end{array}
\right)
\]

\subsubsection{Tabelas}
Para a utilização de Tabelas nas normas da ABNT é preciso usar os comandos:

\begin{lstlisting}
\begin{table}[htb]
 \IBGEtab{%
  \caption{Aqui vai o titulo da tabela}%
  \label{tab_Cap2_exemplo}
 }{%
  \begin{tabular}{ccc}
  	\toprule
  	     Nome       &     Cidade     &    Estado    \\ \midrule\midrule
  	Pedro da Silva  &     Cuiaba     & Mato Grosso  \\
  	  Joao Neves    &   Sao Paulo    &  Sao Paulo   \\
  	Maria Antonieta & Belo Horizonte & Minas Gerais \\ \bottomrule
  \end{tabular}%
 }{%
  \fonte{Autoria Propria}%
  \nota[Nota1]{Aqui pode ser inserido uma nota sobre a tabela}
 }
\end{table}
\end{lstlisting}

Isto fornece a seguinte tabela:
\begin{table}[htb]
	\IBGEtab{%
		\caption{Aqui vai o titulo da tabela}%
		\label{tab_Cap2_exemplo}
	}{%
		\begin{tabular}{ccc}
			\toprule
			     Nome       &     Cidade     &    Estado    \\ \midrule\midrule
			Pedro da Silva  &     Cuiaba     & Mato Grosso  \\
			  João Neves    &   São Paulo    &  São Paulo   \\
			Maria Antonieta & Belo Horizonte & Minas Gerais \\ \bottomrule
		\end{tabular}%
	}{%
		\fonte{Autoria Própria}%
		\nota[Nota1]{Aqui pode ser inserido uma nota sobre a tabela}
	}
\end{table}


\subsection{Apresentação de Códigos de Programação}

\section{Inserção de Imagens}
Para adicionar imagens, conforme as normas da ABNT, é necessário utilizar o código:
\begin{lstlisting}
\begin{figure}[htb]			
  \caption{Brasao Unemat \label{fig_Cap2_brasaoUnemat}}
  \begin{center}
    \includegraphics[scale=0.1]{./Imagens/Brasao_Unemat.png}
  \end{center}
  \legend{Fonte: http://sinop.unemat.br/site/}
\end{figure}
\end{lstlisting}

Com este código, a imagem fica na seguinte forma:

\begin{figure}[htb]			
	\caption{Brasão Unemat \label{fig_Cap2_brasaoUnemat}}
	\begin{center}
		\includegraphics[scale=0.1]{./Imagens/Brasao_Unemat.png}
	\end{center}
	\legend{Fonte: http://sinop.unemat.br/site/}
\end{figure}